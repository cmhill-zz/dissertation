
\renewcommand{\thechapter}{2}
\chapter{Comparing Whole-Genome Assemblies}

\section{Introduction}
% The genome sequence of an organism is a critical resource for
% biologists trying to understand the organism's function and
% evolution.  Obtaining this sequence is difficult as modern sequencing
% technologies can only ``read'' small pieces of the genome (called
% \emph{reads}).  The fact that these tiny \emph{reads} (under a few
% thousands of basepairs in length) can be glued together to reconstruct genomes
% comprising millions to billions of basepairs is by no means evident
% and was the subject of vigorous scientific debate during the early
% days of sequencing technologies~\cite{green1997against,weber1997human}. The modern genomic revolution was in no small part made
% possible by the development of algorithms and computational tools called
% \emph{genome assemblers} able to reconstruct near-complete
% representations of a genome's sequence from the fragmented data
% generated by sequencing instruments.  Despite tremendous advances made
% over the past 30 years in both sequencing technologies and assembly
% algorithms, genome assembly remains a highly difficult computational
% problem.  In all but the simplest cases, genome assemblers cannot
% fully and correctly reconstruct an organism's genome.  Instead, the
% output of an assembler consists of a set of contiguous sequence
% fragments (\emph{contigs}), which can be further ordered and oriented
% into \emph{scaffolds}, representing the relative placement of the
% contigs, with possible intervening gaps, along the genome.

% why assembling is hard?

% Theoretical analyses of the assembly problem, commonly formulated as
% an optimization problem within an appropriately defined graph, have
% shown that assembly is
% NP-hard~\cite{myers1995,medvedev2007computability}, i.e., finding the
% correct optimal solution may require an exhaustive search of an
% exponential number of possible solutions.  The difficulty of genome
% assembly is due to the presence of repeated DNA
% segments (\emph{repeats}) in most genomes. Repeats longer than the length of the sequenced reads lead to ambiguity in the reconstruction of the genome
% -- many different genomes can be built from the same set of
% reads~\cite{nagarajan2009complexity,kingsford2010assembly}.
%
% As a result, practical implementations of assembly algorithms (such as
% ABySS~\cite{ABySS}, Velvet~\cite{Velvet},
% SOAPdenovo~\cite{SOAPdenovo}, etc.) return just an approximate
% solution that either contains errors, or is fragmented, or both.
% Ideally, in a genomic experiment, assembly would be followed by the
% scrupulous manual curation of the assembled sequence to correct the
% hundreds to thousands of errors~\cite{salzberg2005misassemblies}, and
% fill in the gaps between the assembled
% contigs~\cite{nagarajan2010finishing}. Despite the value of fully
% completed and verified genome sequences~\cite{fraser2002value}, the
% substantial effort and associated cost necessary to conduct a
% finishing experiment to its conclusion can only be justified for a
% few high-priority genomes (such as reference strains or model
% organisms). The majority of the genomes sequenced today are
% automatically reconstructed in a ``draft'' state.  Despite the fact
% that valuable biological conclusions can be derived from draft
% sequences~\cite{branscomb2002high}, these genomes are of uncertain
% quality~\cite{chain2009genome}, possibly impacting the conclusions of
% analyses and experiments that rely on their primary sequence.

% Assessing the quality of the sequence output by an assembler is, thus,
% of critical importance, not just to inform downstream analyses, but
% also to allow researchers to choose from among a rapidly increasing
% collection of genome assemblers. Despite apparent incremental
% improvements in the performance of genome assemblers, none of the
% software tools available today outperforms the rest in all assembly
% tasks.  As highlighted by recent assembly
% bake-offs~\cite{earl2011assemblathon,salzberg2011gage}, different
% assemblers ``win the race'' depending on the specific characteristics
% of the sequencing data, the structure of the genome being assembled,
% or the specific needs of the downstream analysis process.
% Furthermore, these recent competitions have highlighted the inherent
% difficulty of assessing the quality of an assembly.  More
% specifically, all assemblers attempt to find a trade-off between
% contiguity (the size of the contigs generated) and accuracy of the
% resulting sequence.  Evaluating this trade-off is difficult even when
% a gold standard is available, e.g., when re-assembling a genome with
% known sequence.  In most practical settings, a reference genome
% sequence is not available, and the validation process must rely on
% other sources of information, such as independently derived data from
% mapping experiments~\cite{zhou2007validation}, or from transcriptome
% sequencing~\cite{adamidi2011novo}. Such data are, however, often not
% generated due to their high cost relative to the rapidly decreasing
% costs of sequencing. Most commonly, validation relies on \emph{de
%   novo} approaches based on the sequencing data alone, which include
% global ``sanity checks'' (such as gene density, expected to be high in
% bacterial genomes, measured, for example, through the fraction of the
% assembled sequence that can be recognized by PFAM
% profiles~\cite{genovo2011}) and internal consistency
% measures~\cite{amosvalidate2008} that evaluate the placement of reads
% and mate-pairs along the assembled sequence.
%
% The validation approaches outlined above can highlight a number of
% inconsistencies or errors in the assembled sequence, information
% valuable as a guide for further validation and refinement experiments,
% but difficult to use in a comparative setting where the goal is to
% compare the quality of multiple assemblies of a same dataset.  For
% example, even if a reference genome sequence is available, while all
% differences between the reassembled genome and the reference are, at
% some level, assembly mistakes, it is unclear whether one should weigh
% single nucleotide differences and short indels as much as larger
% structural errors (e.g., translocation or large scale copy-number
% changes)~\cite{earl2011assemblathon} when comparing different
% assemblies.  Furthermore, while recent advances in visualization
% techniques, such as the FRCurve of Narzisi et
% al.~\cite{FRC2011,vezzi2012feature}, have made it easier for
% scientists to appropriately visualize the overall tradeoff between
% assembly contiguity and correctness, there exist no established
% approaches that allow one to appropriately weigh
% the relative importance of the multitude of assembly quality measures,
% many of which provide redundant information~\cite{vezzi2012feature}.

Here we propose an objective and holistic approach for evaluating and
comparing the quality of assemblies derived from a same dataset.  Our
approach defines the quality of an assembly as the likelihood that the
observed reads are generated from the given assembly, a value which can
be accurately estimated by appropriately modeling the sequencing
process.  This basic idea was formulated in the 1990's in the
pioneering work of Gene Myers~\cite{myers1995}, where he suggested
the correct assembly of a set of reads must be consistent (in terms of
the Kolmogorov-Smirnoff test statistic) with the statistical
characteristics of the data generation process.  The same basic idea
was further used in the arrival-rate statistic (A-statistic) in Celera
assembler~\cite{CeleraAssembler} to identify collapsed repeats, and
as an objective function in quasi-species (ShoRAH~\cite{SHORAH},
ViSpA~\cite{VISPA}), metagenomic (Genovo~\cite{genovo2011}),
general-purpose assemblers~\cite{medvedev2009maximum}, and recent assembly
evaluation frameworks (ALE ~\cite{clark2013ale}, CGAL~\cite{rahman2013cgal}).

In this chapter, we will describe in detail a mathematical model of the
sequencing process that takes into account sequencing errors and
mate-pair information, and show how this model can be computed in
practice. We will also show that this \emph{de novo} probabilistic framework is able to automatically and accurately reproduce the
reference-based ranking of assembly tools produced by the
Assemblathon~\cite{earl2011assemblathon} and GAGE~\cite{salzberg2011gage}
competitions.
Our work is similar in spirit to the recently published ALE~\cite{clark2013ale} and CGAL~\cite{rahman2013cgal}; however, we provide here several extensions of practical importance.  First, we propose and evaluate a sampling-based protocol for computing the assembly score which allows the rapid approximation of assembly quality, enabling the application of our methods to large datasets.  Second, we evaluate the effect of unassembled reads and contaminant DNA on the relative ranking of assemblies according to the likelihood score.
Finally, we will
demonstrate the use of our probabilistic quality measure as an
objective function in optimizing the parameters of assembly programs.
The software implementing our approach is made available, open-source
and free of charge, at: \url{http://assembly-eval.sourceforge.net/}.


%%%%%%%%%%%%%%%%
%% Theory %%
%%
\section{Methods}
\label{theory}

\subsection{Theoretical foundation for probabilistic evaluation}
In this section, we formalize the probabilistic formulation of
assembly quality and the model of the sequencing process that
allows us to compute the likelihood of any particular assembly of a
set of reads.  We will show that the proposed probabilistic score is
correct in the sense that the score is maximized by the true genome
sequence.

\subsubsection{Likelihood of an assembly}
Let $A$ denote the event that a given assembly is the true genome
sequence, and let $R$ denote the event of observing a given set of
reads.  In the following, we will use the same symbol to denote the
assembly sequence and the event of observing the assembly.  We will
also use the same symbol to denote the set of reads and the event of
observing the set of reads.

According to Bayes' rule, given the observed set of reads, the probability of the assembly can be written as:

\begin{equation}
  \Pr[A \vert R] = \frac{\Pr[R \vert A] \Pr[A]}{\Pr[R]}
\end{equation}

\noindent where $\Pr[A]$ is the \emph{prior probability} of observing the genome
sequence $A$. Any prior knowledge about the genome being assembled
(e.g., approximate length, presence of certain genes, etc.) can be
included in $\Pr[A]$; however, for the purpose of this paper, we will
assume that this prior probability is constant across the set of
``reasonable'' assemblies of a same set of reads.
Given commonly available information about the genomes, formulating a precise mathematical framework for defining $\Pr[A]$ is an extensive endeavor beyond the scope of this paper.

Similarly, $\Pr[R]$ is the prior probability of observing the set of
reads $R$.  Since our primary goal is to compare multiple assemblies
of a same set of reads, rather than to obtain a universally accurate
measure of assembly quality, we can assume $\Pr[R]$ is a constant as
well.  Thus, for the purpose of comparing assemblies, the values $\Pr[A
  \vert R]$ and $\Pr[R \vert A]$ are equivalent.  The latter, the
posterior probability of a set of reads, given a particular assembly of
the data, can be easily computed on the basis of an appropriately
defined model of the sequencing process and will be used in our paper
as a proxy for assembly quality.

Under the assumption that individual reads are independent of each
other (violations of this assumptions in the case of mate-pair
experiments will be discussed later in this section), $\Pr[R \vert A]
= \prod_{r \in R} \Pr[r \vert A]$. If the set of reads is unordered, we need to account for the different permutations that generate the same set of reads.  As this value is a constant for any given set of reads, we ignore it in the rest of our paper.

$\Pr[r \vert A]$, hereafter referred to as $p_r$, can be computed using an appropriate
model for the sequencing process.  Throughout the remainder of the
paper, we will discuss increasingly complex models and their impact on
the accuracy of the likelihood score.

\subsubsection{True genome obtains the maximum likelihood}
\label{maximizing_likelihood}
Any useful assembly quality metric must achieve its maximum value when
evaluating the true genome sequence; otherwise, incorrect assemblies of
the data would be preferred.  We prove below that the likelihood
measure proposed in our paper satisfies this property.

\edit{Assuming that we have a set of reads $R$ from the true genome, produced by generating exactly one single-end read from each location in the genome without errors and with a fixed length.  Given the set of reads $R$, the probability a particular read is generated from the true genome is precisely the number of times the read occurs in $R$ divided by the size of $R$ (note that multiple reads can have the same sequence, e.g., when generated from repeats). Let $N_s$ denote number of times that the sequence $s$ occurs in $R$, and $q_s = N_s/|R|$ denote the probability that sequence $s$ is generated from the true genome.  To show that the true genome maximizes the likelihood score, let us assume that we have some assembly $A$ and $p_s$ is the probability that the sequence $s$ is generated from the assembly $A$.}

\edit{Given assembly $A$, our likelihood score is then the product of ${p_s}^{N_s}$ over all sequences $s$ in $S$, which can be rewritten as $\prod_{s \in S} {p_s}^{q_s|R|} = (\prod_{s \in S} {p_s}^{q_s})^{|R|}$.  Now, note that since $|R|$ is a fixed constant, maximizing the likelihood score is equivalent to maximizing}
\begin{align*}
\prod_{s \in S} {p_s}^{q_s}
\end{align*}

%If $p_s$ and $q_s$ are probability distributions over the set of sequences $S$, then one can show that this expression is maximized when $p_s = q_s$ for all sequences $s \in S$.

%This can be proven by writing the log of the previous expression in terms of the Kullback-Leibler divergence and utilizing its properties.  Note $\log (\prod_{s \in S} {p_s}^{q_s}) = \sum_{s \in S} q_s \log p_s = \sum_{s \in S} q_s \log (p_s/q_s) + \sum_{s \in S} q_s \log q_s = - D_{KL}(Q||P) - H(Q)$, where $D_{KL}(Q||P)$ is the KL-divergence for distributions $Q = \{q_s | s \in S \}$ and $P = \{p_s | s \in S\}$, and $H(Q)$ is the Shannon entropy of $Q$.  Since the KL-divergence is always positive and only equal to 0 when $Q = P$ (i.e. $q_s = p_s$ for all $s \in S$), then we can conclude that the likelihood score is maximized when $q_s = p_s$ for all $s \in S$.  To complete the proof, note that if the assembly $A$ is the true genome, then the probabilities $p_s$ associated with $A$ are precisely equal to $q_s$ for all sequences $s \in S$, so the likelihood score is maximized by the true genome.

%Any useful assembly quality metric must achieve its maximum value when
%evaluating the true genome sequence; otherwise, incorrect assemblies of
%the data would be preferred.  We prove below that the likelihood
%measure proposed in our paper satisfies this property.

%Let $Q$ denote the probability distribution of observing reads from
%the true genome.  As the size of the read set grows, the distribution
%of the observed reads approaches $Q$.  Let $P$ denote the probability
%distribution of observing reads from a specific assembly of the same
%%data.  $P$ can be calculated given the sequence of the assembly and
%the reads.

%The average probability of the set of observed reads being generated from an assembly is
%\begin{align*}
%  \prod_{s \in \Sigma^*} {p_s}^{q_s},
%\end{align*}

%\noindent where $p_s$ and $q_s$ are the probabilities of the sequence $s$ according to the observed and true distributions, respectively.
The likelihood can be re-written as
\begin{align*}
\log (\prod_{s \in S} {p_s}^{q_s}) & = \sum_{s \in S} q_s \log p_s \\
& = \sum_{s \in S} q_s \log (\frac{p_s}{q_s}) + \sum_{s \in S} q_s \log q_s \\
& = - D_{KL}(Q||P) - H(Q),
\end{align*}

%\log (\prod_{s \in \Sigma^*} {p_s}^{q_s}) & = \sum_{s \in \Sigma^*} q_s \log p_s \\
%& = \sum_{s \in \Sigma^*} q_s \log (\frac{p_s}{q_s}) + \sum_{s \in \Sigma^*} q_s \log q_s \\
%& = - D_{KL}(Q||P) - H(Q),

\noindent where $D_{KL}(Q||P)$ is the KL-divergence for the distributions $Q$ and
$P$, and $H(Q)$ is the Shannon entropy of $Q$.  Since the
KL-divergence is always non-negative and only equal to 0 if and only if $Q = P$,
the average probability is maximized if the assembly is equal to the
true genome.

Even though the true genome does maximize the likelihood in this
model, there may be other assemblies that achieve the same optimal
score as long as these assemblies yield probabilities $p_s$ which are
equal to the probabilities $q_s$ for every sequence $s$.  This can happen, for example,
in the case of a misassembly that is nonetheless consistent with the
generated reads.  This situation highlights the loss of information
inherent in modern sequencing experiments -- without additional
long-range information, the information provided by the reads
themselves is insufficient to distinguish between multiple possible
reconstructions of a genome~\cite{kingsford2010assembly}.

\subsubsection{Error-free model for fragment sequencing}
\label{error_free_model}
The most basic model for the sequencing process is the
\emph{error-free model}. In this model, we assume reads of a given
fixed length (a more general read length distribution can be included
in the model but would not impact comparative analyses of assemblies
derived from a same set of reads). We further assume that reads are
uniformly sampled across the genome, i.e., that every position of the
genome is equally likely to be a starting point for a read. This
simplifying assumption is made by virtually all other theoretical
models of genome assembly, despite the biases inherent to all modern
sequencing technologies.  A more accurate, technology-dependent, model
can be obtained by including additional factors that account, for
example, for DNA composition biases.  For the purpose of generality,
we restrict our discussion to the uniform sampling model.
Furthermore, for the sake of simplicity, we assume (1) that the true
genome consists of a single circular contiguous sequence, (2) that our
assembly is also a single contig, and (3) that every read can be
mapped to the assembly.  We will later discuss extensions of our model
that relax these assumptions.

Under these assumptions, we can compute the probability of a
read $r$ given the assembled sequence as:
\begin{equation}
  \label{error_free_probability}
  p_r = \frac{n_r}{2L}
\end{equation}
where $n_r$ represents the number of places where the read occurs in
the assembled sequence of length $L$.  The factor $2$ is due to the
fact that reads are sampled with equal likelihood from both the
forward and reverse strands of a DNA molecule.  This formulation was
previously used by Medvedev \emph{et al.}~\cite{medvedev2009maximum}
to define an objective function for genome assembly.

\subsection{A realistic model of the sequencing process}
The error-free model outlined above makes many simplifying assumptions
that are not representative of real datasets.  Here we demonstrate
how the model can be extended to account for artifacts such as
sequencing errors, mate-pair information, assemblies consisting of
multiple contigs, and the presence of un-mappable reads.

\subsubsection{Sequencing errors}
\label{methods_errors}
All current technologies for sequencing DNA have a small but
significant probability of error.  Here we focus on three common types
of errors: the insertion, deletion, and substitution of a nucleotide.

In the error-free model, the probability of a read having been
generated from a position $j$ in the sequence is one if the read
exactly matches the reference at that position and zero otherwise.  We
now extend this model such that the probability of each read having
been generated from any position $j$ of the reference is a real value
between zero and one, representing the likelihood that a sequencing
instrument would have generated that specific read from that specific
position of the reference.  This value clearly depends on the number of differences
between the sequence of the read and the sequence of the reference at
position $j$.  Given the assembled sequence, the probability of a particular read will be the cumulative probability of the read
across all possible locations in the genome.

Specifically, let us denote the probability that read $r$ is observed
by sequencing the reference, \emph{ending} at position $j$ by
$p_{r,j}$.  Then, the total probability of the read $r$ is
\begin{align}
\label{prob_error_sum}
p_r = \frac{{\sum_{j=1}^{L} p^{\text{forward}}_{r,j}} + {\sum_{j = 1}^{L} p^{\text{reverse}}_{r,j}}}{2L}
\end{align}

The individual probabilities $p_{r,j}$ can be computed if we
do not model insertion and
deletion errors and only allow substitution errors which occur with probability $\epsilon$.  The per-base probability of a substitution error can be set individually for each based on the quality value
produced by the sequencing instrument. Then, $p_{r, j} = \epsilon^{s}(1 - \epsilon)^{l - s}$
, where $s$ is the number of substitutions needed to match read $r$ to
position $j$ of the reference sequence.
In the more general case,
$p_{r,j}$ values can be computed using dynamic programming.

\subsubsection{Exact probability calculation via dynamic programming}
\label{methods_dynamic}
For a model of the sequencing process that allows insertions, deletions,
and substitutions with specific probabilities, we can exactly compute
probability, $p_r = \Pr[r \vert A]$, of observing a read $r$ given an
assembly $A$ using a dynamic programming algorithm.  In general, we
want to find the sum of the probabilities of all possible alignments of a
read to a position of the assembly.


\begin{figure}[h!]
\begin{center}
\includegraphics[]{figures/fig_1_optimal_alignments}
\end{center}
\renewcommand{\baselinestretch}{1}
\small\normalsize
\begin{quote}
\caption[Multiple optimal read alignments.]{Two different optimal alignments of the read \textbf{ACG} to the assembly \textbf{ACCG}. Our dynamic programming algorithm finds the sum of the probabilities of all possible alignments.}
\label{different_optimal_alignments}
\end{quote}
\end{figure}
\renewcommand{\baselinestretch}{2}
\small\normalsize
%\newpage

% \begin{figure}
% \begin{center}
% \includegraphics[width=4.572in]{metagenome}
% \end{center}
% \renewcommand{\baselinestretch}{1}
% \small\normalsize
% \begin{quote}
% \caption[The metagenome of an environment]{The metagenome of an environment can be viewed as the concatenation of the organisms found in the environment whose multiplicity is determined by their abundance. \label{fig:metagenome}}
% \end{quote}
% \end{figure}
% \renewcommand{\baselinestretch}{2}
% \small\normalsize
% \newpage


% \begin{figure}[h!]%tb
% \begin{center}
% \includegraphics[]{figures/fig_1_optimal_alignments}
% \end{center}
% \caption{Multiple optimal read alignments. Two different optimal alignments of the read \textbf{ACG} to the assembly \textbf{ACCG}. Our dynamic programming algorithm finds the sum of the probabilities of all possible alignments.}
% \label{different_optimal_alignments}
% \end{figure}

The number of such possible alignments grows exponentially as a
function of read length.  Most of those alignments have a very
small probability.  However, several alignments may have probabilities
that are equal or close to the optimal.  For example, the two alignments
of the same pair of sequences in
Figure~\ref{different_optimal_alignments} have the same probability
and are both optimal alignments.

We use a dynamic programming algorithm (similar to the ``forward''
algorithm in Hidden Markov Models) to efficiently calculate the sum of the
probabilities of all alignments of a read to the assembly as follows.
In the formula~\eqref{prob_error_sum}, $p^{\text{forward}}_{r,j}$ and
$p^{\text{reverse}}_{r,j}$ are the sum of the probabilities
of \emph{all} possible alignments of the read $r$ to, respectively, the
reference and its reverse complement, ending at position $j$.

We define $T[x,y]$ as the probability of observing prefix $[1 \ldots
y]$ of the read $r$, if $y$ bases are sequenced from the reference, ending
at position $x$.  Therefore, $p_{r, j} = T[j, l]$. $T[x, 0]$ represents the probability of observing an empty sequence if
we sequence zero bases and is set to 1.  $T[0, y]$
represents the probability of observing prefix $[1 \ldots y]$ of the
read if $y$ bases are sequenced from the reference, ending at position
$0$ (before the beginning), and is set to 0.

For $x \geq 1$ and $y \geq 1$, $T[x, y]$ is recursively defined:

\begin{align}
\label{dp_single}
  T[x, y] = & \quad T[x - 1, y - 1] \Pr[\operatorname{Substitute}(A[x], r[y])]  \\
  & + T[x, y - 1]  \Pr[\operatorname{Insert}(r[y])]  \notag\\
  & + T[x - 1, y]  \Pr[\operatorname{Delete}(A[x])],  \notag
\end{align}

\noindent where $r[y]$ and $A[x]$ represent the nucleotides at positions $y$ and
$x$ of the read $r$ and the assembly $A$, respectively.
$\Pr[\operatorname{Substitute}(A[x], r[y])]$ is the probability of
observing the nucleotide $r[y]$ by sequencing the nucleotide $A[x]$.
In our experiments, we did not distinguish between
different types of errors and considered their probability to be
$\epsilon$ and the probability of observing the correct nucleotide to
be $1 - \epsilon$.

The dynamic programming algorithm outlined above has a running time of
$O(lL)$ per read.  Even though the running time is polynomial, it is
slow in practice.  However, we can speed it up by using alignment
seeds.  The seeds would give us the regions of the assembly where a
read may align with high probability.  We can apply the dynamic
programming only to those regions and get a very good approximate
value of the total probability.  We use exact seeds ($k$-mers) of a
given length to build a hash index of the assembly sequence.  Then,
each read is compared to the regions where it has a common $k$-mer
with the assembly sequence.

\subsubsection{Mate pairs}
Many of the current sequencing technologies produce paired reads --
reads generated from the opposite ends of the same DNA fragment.  This
information is extremely valuable in resolving genomic repeats and in
ordering the contigs along long-range scaffolds; however, the paired
reads violate the assumption that reads are sampled independently,
that we made in the discussion above.  To address this issue, we can use the
pairs rather than the individual reads as the underlying objects from
which the assembly likelihood is computed.  To address the possibility
that assembly errors may result in violations of the constraints imposed by
the paired reads, we only consider pairs for which both ends align to
a same contig or scaffold within the constraints imposed by the
parameters of the sequencing experiment.  Any pairs that violate these
constraints get classified as unassembled.  Note that in addition to
sequencing errors, we now also handle fragment sizing errors --
deviations of the estimated distance between paired reads from the
distance implied by the sequencing experiment.  We model the
distribution of fragment sizes within a same library by a normal
distribution, using user-supplied parameters, and use this information
to appropriately scale the likelihood estimate for each possible
placement of a mate pair along the genome.

We modify the dynamic programming recurrence from formula~\eqref{dp_single} to handle the probability calculation for the paired reads as follows.
The probability of the first read in the pair is calculated as the same as in the formula~\eqref{dp_single}.  For the second read, we adjust the dynamic programming to ensure that it is aligned within a certain distance downstream of the alignment of the first read.
We modify the first column of the dynamic programming table of the \emph{second} read in the pair to take into account the distance from the first read.

Formally, given a paired read, we define $T_2[x,y]$ as the probability of observing prefix $[1 \ldots y]$ of the $2$nd read in the pair, if $y$ bases are sequenced from the reference, ending at position $x$.
%Given paired reads $1$ and $2$, we define $T_1[x,y]$ and $T_2[x,y]$ as the probabilities of observing prefixes $[1 \ldots y]$ of reads $1$ and $2$, if $y$ bases are sequenced from the reference ending at position $x$ from reads $1$ and $2$, respectively.
Assume that the second read occurs after the first read and is separated by a normally-distributed distance with mean $\mu$ and with a standard deviation $\sigma$.
%, then all of read $1$ must be aligned before read $2$.
Therefore,

\begin{align}
T_2[x, 0] = \sum_{i=1}^{x}{\Pr[\operatorname{insert}(x-i)|N(\mu,\sigma)))] + T_1[x-i,l]},
\end{align}

\noindent where $\Pr[\operatorname{insert}(n)|N(\mu,\sigma)))]$ is the probability of observing an insert size of length $n$ from a normal distribution with parameters $\mu$ and $\sigma$, and $l$ is the length of the first read in the pair.

Instead of using two tables, we can concatenate the read pair together with a special character ($M$), which will signal when the insert size should be taken into account.
%Let $r$ represent a concatenation of paired reads with a special character $M$ separating them.
%Let $r[y]$ represent the nucleotide at position $y$ of the paired read $r$.
%Let $\Pr[\operatorname{insert}(n)|\mu,\sigma))]$ be the probability obtaining an insert size of length $n$ from a normal distribution with parameters $\mu$ and $\sigma$.

For $x \geq 1$ and $y \geq 1$, $T[x, y]$ is recursively defined as follows:

\begin{equation}
\begin{aligned}
T[x, y] = \quad
\text{if }r[y] == M&\begin{cases}
\sum_{i=1}^{x}{\Pr[\operatorname{insert}(x-i)|N(\mu,\sigma)))] + T[x-i,y-1]}
\end{cases}
\\
\text{else }&\begin{cases} T[x - 1, y - 1] \Pr[\operatorname{Substitute}(A[x], r[y])] \\
\quad + T[x, y - 1]\Pr[\operatorname{Insert}(r[y])]  \\
\quad + T[x - 1, y]\Pr[\operatorname{Delete}(A[x])]
\end{cases}
\end{aligned}
\end{equation}

\subsubsection{Assemblies containing more than one contig}
As we mentioned in the introduction, the output of an assembler
usually consists of a (large) set of contigs rather than one single
contig, representing the genome being assembled.
In the extreme case, an ``assembler'' may return the set of
unassembled input reads (\edit{or the set of all k-mers in De Bruijn-based assemblers}) as its output.  Our likelihood score must be modified to account for such fragmented assemblies.

In practice, most assemblers join contigs only if they overlap by more than a certain number of bases; however, we only consider the case where contigs are non-overlapping substrings of the true genome.
In this case, the length of the original genome must be \emph{at least} the
sum of the lengths of the contigs, that is, $\sum L_j$, where $L_j$ is the
length of the $j$th contig. Therefore, the probability of every read
is at most:

\begin{equation}
\begin{aligned}
  \frac{n_r}{2\sum L_j}
\end{aligned}
\end{equation}

Overlapping contigs can be handled by reducing the length of the contigs by a value representing the minimum overlap required by the assembler, as performed, for example, in Genovo~\cite{genovo2011}.

%In practice, most assemblers join contigs only if they overlap by more
%than a certain number of bases, i.e., distinct contigs may overlap.
%Assuming contigs are allowed to overlap by less than $o$ bases
%(minimum overlap necessary to merge adjacent contigs), the length of
%the genome from which the reads were sampled must be greater than or
%equal to $\sum (L_j - o)$. Thus, the probability of a read is
%estimated by:
%\begin{align*}
%  \frac{n_r}{2\sum (L_j - o)}
%\end{align*}

%{\bf HOW DO THESE CHANGE IN THE FORWARD ALGORITHM FORMULATION?}

\subsubsection{Reads that do not align well}
\label{methods_practical_unassembled}
In practice, popular assemblers do not incorporate every read in the
assembly.  Possible reasons include assembly errors (such as collapsed
tandem repeats), reads with high error rates, or contamination in the
DNA sample. These ``singleton'' or ``chaff'' reads cannot be modeled
by our likelihood approach as the likelihood of any assembly that
does not incorporate every read is zero.
When sequencing errors are modeled, every read obtains a non-zero likelihood, even if it does not align to the assembly. Since, in general, a non-trivial fraction of the total set of the reads cannot be mapped to the assembly, by their sheer number, the singleton reads would dominate the probability calculation.

To account for this factor, we argue that for any read that does not
align well, the overall probability of the assembly should not be
lower than the probability of the same assembly when the missing read
is appended to its sequence as a separate contig. The effect of such an addition on the
overall probability can be calculated as follows.  First, the
probability of observing this read exactly,
$\left(\frac{\Pr[\text{exact match}]}{2L}\right)$, is multiplied to
the product of the probabilities of all mapped reads.  Second, the
probabilities of the mapped reads are decreased slightly due to the
increase in the length of the assembled sequence.

For simplicity, let us assume an error-free model where each read maps
to exactly one position on the assembled sequence.  Let $k$ denote the
number of the original reads. The ratio between the new probability
for all original reads divided by their probability before adding the
new read is:
\[
  \frac{\frac{1}{(L + l)^k}}{\frac{1}{L^k}} = \left(\frac{L}{L + l}\right)^k = \left(1-\frac{l}{L + l}\right)^k \approx e^{-\frac{lk}{L}}
\]

Therefore, if the probability of observing a read is less than
\begin{equation}
  \label{probability_threshold}
  \frac{\Pr[\text{exact match}]}{2L}e^{-\frac{l\left\vert R\right\vert}{L}},
\end{equation}
we consider this read as ``unmapped'' and
use formula~\eqref{probability_threshold} as its probability.  The probability
of an exact match $\Pr[\text{exact match}]$ is approximated by
$\left(1 - \epsilon\right)^{l}$, where $\epsilon$ is the probability of
an error (a mismatch, an insertion, or a deletion).

\subsection{Performance considerations}
\subsubsection{Estimating the average read likelihood by sampling}
\label{sampling_reads}
Depending on the specific characteristics of the chosen sequencing model, the computation of the probability $\Pr[R \vert A]$ can be expensive
for the dataset sizes commonly encountered in current projects (tens
to hundreds of millions of reads).  In such cases, we can approximate
the likelihood of an assembly by using a random subset of the reads
$R^\prime \subseteq R$. To counteract the effect of the size of the
sample on the computed probability, we define the assembly quality as
the geometric mean of the individual read probabilities:
\begin{equation}
  \label{average_probability}
  \operatorname{AP}(R^\prime) =
  \left(\prod_{r \in R^\prime} p_r\right)^{\frac{1}{\left|R^\prime\right|}}
\end{equation}

The logarithm of this value (Log Average Probability (LAP)) is
reported in the remainder of the paper as the assembly quality ``score'':
\begin{equation}
  \label{average_log_probability}
  \operatorname{LAP}(R^\prime) = \log_{10} \left( \operatorname{AP}(R^\prime) \right) = \frac{\sum_{r \in R^\prime} \log_{10} p_r}{\left|R^\prime\right|}
\end{equation}

In other words, we define the assembly quality as the average log
likelihood of the reads given an assembly.  This formulation
also allows us to estimate the accuracy of the approximate likelihood
value produced by sub-sampling the set of reads.  According to
sampling theory, the distribution of the scores across multiple
samples has the mean equal to the true likelihood of the assembly
(computed from all the reads) and a standard error proportional to
$\frac{1}{\sqrt{\left|R^\prime\right|}}$, i.e., the larger the sample is,
the more accurate our estimation is for the likelihood of the true assembly.
Since the probability of a read is bounded by formula (\ref{probability_threshold}), the variance of the sample can also be bounded by this value.

In practice, we increase the sample size until the assemblies can be unambiguously distinguished by the LAP value.
Specifically, we increase the sample size, by binary search, until the LAP values are separated by at least a single standard deviation.  The level of subsampling required will, thus, be dependent on the extent of the differences between the assemblies –- for very different assemblies, low levels of subsampling are sufficient.

\subsubsection{Approximating the likelihood value using an aligner}
\label{methods_aligner}
Alternatively, when it is impractical to calculate exact probabilities
for large sets of reads, we can approximate these probabilities using
fast and memory-efficient alignment search programs, which internally
model the sequencing process.  We use Bowtie 2~\cite{langmead2012fast} to align the reads to
the assembly.  However, our programs are easy to adapt for any
read alignment tool that stores the alignment results in
SAM\cite{li2009sequence} format.

For each reported alignment, we use the number of substitutions $s$ to
compute the probability $p_{r}$.  The probability of this alignment, $p_{r,j}$,
can be approximated by $\epsilon^{s}(1 - \epsilon)^{l - s}$ and

\begin{equation}
\label{}
p_{r} = \frac{\sum_{j \in S_r} p_{r,j}}{2L},
\end{equation}
where $S_r$ is the set of alignments in the SAM file for the read $r$.

We can further extend this equation to mated reads.  A pair of mated
reads aligns if the distance and orientation of the alignment of the
pair are consistent with the experimental design parameters.  Given
read $i_1$ and its mate $i_2$, we compute $p_{(i_1,i_2)}$ by
multiplying the probabilities of individually aligning each mate at
their respective positions with the probability that they are
separated by their distance from each other. That is,
\begin{equation}
\label{mate_pair_prob}
p_{(i_1,i_2)} = \frac{\sum_{(j_1,j_2) \in S_{(i_1,i_2)}} p_{i_1,j_1} p_{i_2,j_2} \Pr[\textrm{insert}(j_2 - (j_1 + l_1))]}{2(L - l)},
\end{equation}
where $p_{i_1, j_1} = \epsilon^{s_1}(1 - \epsilon)^{l_1 - s_1}$.  Mate
pair insert sizes follow a normal distribution with mean and
standard deviation being estimated from the parameters of the sequencing process.  Unless otherwise stated, the standard deviation is 10\%
of the insert size.  If only one of the mates, $i_1$ or $i_2$, maps,
the probability $p_{(i_1,i_2)}$ is $0$. We use (\ref{probability_threshold}) to set the probability for this case.

In our experiments, Bowtie 2 was used to approximate the read
probabilities for the larger datasets; however, it could be substituted with any other aligner.

%%%%%%%%%%%%%%%%
%% Methods %%
%%

\subsection{Datasets}

The read data for \emph{Rhodobacter sphaeroides} 2.4.1 was downloaded from \url{
http://gage.cbcb.umd.edu/data/Rhodobacter_sphaeroides}, and the
corresponding reference sequence was obtained from the NCBI RefSeq database (NC\_007493.1, NC\_007494.1, NC\_009007.1, NC\_007488.1,
NC\_007489.1, NC\_007490.1, NC\_009008.1).
In addition, two more \emph{Rhodobacter} genomes were selected as
reference genomes, specifically \emph{R. sphaeroides} ATCC 17025 (NCBI
IDs NC\_009428.1, NC\_009429.1, NC\_009430.1, NC\_009431.1,
NC\_009432.1), and \emph{R. capsulatus} SB1003 (NC\_014034.1,
NC\_014035.1).

The read data for \emph{Stapylococcus aureus} USA300 was downloaded from \url{
http://http://gage.cbcb.umd.edu/data/Staphylococcus_aureus}, and the
corresponding reference sequence was obtained from the NCBI RefSeq database (NC\_010063.1, NC\_010079.1, NC\_012417.1).
In addition, two more \emph{Stapylococcus} genomes were selected as
reference genomes, specifically \emph{S. aureus} 04-02981 (CP001844), and \emph{S. epidermidis} ATCC 12228 (AE015929, AE015930,
AE015931, AE015932, AE015933, AE015934, AE015935).

The read data for human chromosome 14 was downloaded from \url{http://gage.cbcb.umd.edu/data/Hg_chr14/}, and the corresponding reference sequence was obtained from the NCBI RefSeq database (NC\_000014.8).

The Assemblathon~1 competition evaluates assemblies on the simulated
short read dataset generated from the simulated 110 Mbp diploid genome.
The competition provides sequence libraries with varying insert sizes (200-10,000 bp)
and coverage (20-40x).
Assemblathon~1 allowed teams to submit multiple entries, but for our
analyses, we only examine the top ranking assemblies from each
team. The raw reads and the consensus sequence of the top ranking
assemblies were downloaded from \url{http://korflab.ucdavis.edu/Datasets/Assemblathon/Assemblathon1/}.

Also used in our analyses is the \emph{E. coli} K12 MG1655 dataset, generated using Illumina MiSeq technology
(300 bp insert size, 370x coverage) ({\url{http://www.illumina.com/systems/miseq/scientific_data.ilmn}}).

%%%%%%%%%%%%%%%%
%% Results %%
%%
\section{Results}

\subsection{Performance-related approximations do not significantly
  affect the likelihood score}

The full and exact computation of the assembly likelihood score is
computationally intensive and ultimately impractical for the analysis
of large genomes sequenced with the next generation technologies. We have
highlighted in the Methods section several approaches that can be used
to reduce the computational requirements and allow the application of
our methods in practical settings, including the computation of the
likelihood score on the subsets of the original set of reads and the
approximation of the score from the output of an alignment program.
As we will show below,
our approximations do not affect the comparative ranking of the multiple assemblies derived from a same dataset.
%these approximations do not significantly
%affect the computed scores, at least in the context of comparing
%multiple assemblies derived from a same dataset.

\subsubsection{The likelihood score is robust under sampling.} To assess the
effect of subsampling, we relied on a collection of the assemblies of the
human chromosome 14 made available by the GAGE assembly `bake-off'.
We sampled random subsets of increasing size (one trial per
size) from the over 60 million reads and
computed the likelihood score based only on the sampled reads.

\begin{figure}[h!]
\begin{center}
\includegraphics[width=4.86in]{figures/fig_2_hg_mean_samples}
\end{center}
\renewcommand{\baselinestretch}{1}
\small\normalsize
\begin{quote}
\caption[LAP-based evaluation of the assemblies for the Human
  chromosome 14 via sampling.]{LAP-based evaluation of the assemblies for the Human
    chromosome 14 via sampling.
  The $x$-axis represents the number of
  sampled reads.  For each assembly, we plot the corresponding LAP
  on a chosen subsample along with the standard deviation.  The relative
  ranking of assemblies becomes fixed with 10,000 reads, which is
  less than 0.02\% of the original reads.}
\label{hg_mean_samples}
\end{quote}
\end{figure}
\renewcommand{\baselinestretch}{2}
\small\normalsize

As seen in Figure~\ref{hg_mean_samples}, the overall ranking of the
different assemblies stabilizes after sampling just
10,000 reads, i.e., less than 0.02\% of the entire dataset. After
this point, the scores of individual assemblies differ by more than
the standard deviation of the sub-sampled scores, indicating the relative
ranking of the assemblies can be determined with high statistical
confidence. This result suggests a practical strategy for computing
the assembly likelihood wherein datasets of increasing size are
repeatedly sampled from the set of reads until the likelihood scores
of the compared assemblies can be distinguished from each
other. The search for the appropriate sample size can start from a
reasonable `guess' (e.g., 0.05\% of the total set of reads), which is
then iteratively doubled until the likelihood scores are separated from
each other by a given multiple of the sampling-induced standard deviation.

\subsubsection{Aligner-based approximation correlates with the
  dynamic-programming computation of the likelihood score.} As outlined in
the Methods section, we relied on an alignment program (in our case,
Bowtie 2~\cite{langmead2012fast}) to estimate the likelihood of individual reads
based on their alignment along the assembly.  This approach is
substantially faster than the more accurate dynamic programming
algorithm that computes the cumulative likelihood of all possible
alignments of a read against the assembly.

\begin{figure}[h!]
\begin{center}
\includegraphics[width=4.86in]{figures/fig_3_aligner_vs_dynamic_sa}
\end{center}
\renewcommand{\baselinestretch}{1}
\small\normalsize
\begin{quote}
\caption[Comparison of the read probability calculation methods]{Comparison of the read probability calculation methods for
    \emph{S. aureus} with 4,788,174 reads. Each mark on the plot
    represents a single read.  The read's position is determined by
    the probability calculated from our dynamic programming method (y-axis)
    and Bowtie 2 (x-axis).  Points on the line $y=x$ denote reads that were
    given the same probability by both methods.  Since Bowtie 2 only
    finds the best alignment, it usually reports a slightly lower
    probability.  A probability threshold of 1e-30 is shown for the
    dynamic programming method.  The read probabilities that fall below
    this threshold would be rounded up to 1e-30 during LAP computation.}
\label{fig:aligner_vs_dynamic}
\end{quote}
\end{figure}
\renewcommand{\baselinestretch}{2}
\small\normalsize

Figure~\ref{fig:aligner_vs_dynamic} compares the per-read likelihood
values with respect to the complete genome sequence of
\emph{Staphylococcus aureus}, using data provided by the GAGE
competition.  In this plot, each read is represented by a point whose
coordinates represent the corresponding likelihood scores computed
through full dynamic programming (y axis) and from Bowtie 2 alignments
(x axis). As the full dynamic programming approach sums over all
possible alignments, the corresponding likelihood values are higher
(points occur above the diagonal) than those estimated by Bowtie 2.
The difference between the two methods becomes less noticeable as the
likelihood increases as more of the probability mass is concentrated
around the best alignment of a read to the reference.

\subsubsection{The likelihood scores correlate with reference-based
  validation}

The recent assembly competitions GAGE~\cite{salzberg2011gage} and Assemblathon~1
\cite{earl2011assemblathon} relied on a combination of \emph{de novo} and
reference-based metrics to compare and rank different assemblies. For
the majority of these datasets, a complete or high-quality draft
sequence was available, allowing the authors to objectively determine
all the errors in the assemblies by aligning them to the reference
sequences. Based on this information, the GAGE and Assemblathon 1
teams proposed several assembly quality metrics that simultaneously
capture some aspects of the contiguity and correctness of an assembly.  Here we
compare our \emph{de novo} likelihood score to these reference-based
metrics.

Generally, the \emph{de novo} LAP scores agree with the reference-corrected contiguity
values (see Tables~\ref{tab:rhodobacter},~\ref{tab:staph}, and ~\ref{tab:rhodobacter}).  Furthermore, the reference genome assembly (assumed to be the
most correct reconstruction of the genome being analyzed) achieves the
highest LAP score while the references derived from the closely-related
organisms are considerably worse than all the other assemblies.  In
other words, the \emph{de novo} LAP scores accurately capture the
relative quality of the different assemblies.

It is important to note that there are several exceptions to
these general observations. In the case of \emph{S. aureus} USA300 (Table~2),
the read-based LAP scores for the Abyss assembly (computed on both
contigs and scaffolds) are better than those obtained for the
reference genome, contradicting our intuition, since ABySS's reference-corrected contiguity is
worse. This result highlights the importance of accurately modeling
the sequencing experiment when computing the LAP scores.  Once
mate-pair information is taken into account, the LAP scores correctly
identify the best assembly.  This phenomenon is due to the fact that
the Abyss assembly is able to incorporate more of the reads however
their placement in the assembly is inconsistent with the mate-pair
linkage information.


\begin{landscape}
\renewcommand{\baselinestretch}{1}
\small\normalsize
\begin{table}[tb!]
\centering
\tiny
\begin{tabular}{|c|c|c|c|c|c|c|c|c|c|c|}
\hline
 & \multicolumn{4}{c|}{Contigs} & \multicolumn{4}{c|}{Scaffolds} & & \\
\hline
Assembler & LAP reads & LAP mates & N50 (kb) & CN50 (kb) & LAP reads & LAP mates & N50 (kb) & CN50 (kb) & Unaligned reads (frac) & Unaligned mates (frac) \\
\hline
ABySS & -20.924 & -27.365 & 5.9 & 4.2 & -20.929 & -27.320 & 9 & 5 & 0.228 & 0.524\\
Allpaths-LG & {\bf -20.795} & {\bf -27.141} & 42.5 & {\bf 34.4} & {\bf -20.796} & {\bf -27.099} & {\bf 3,192} & {\bf 3,192} & {\bf 0.212} & {\bf 0.441} \\
Bambus2 & -21.528 & -27.439 & 93.2 & 12.8 & -21.531 & -27.424 & 2,439 & 2,419 & 0.270 & 0.501\\
CABOG & -22.550 & -27.749 & 20.2 & 17.9 & -22.550 & -27.714 & 66 & 55 & 0.345 & 0.540\\
MSR-CA & -21.496 & -27.407 & 22.1 & 19.1 & -21.497 & -27.324 & 2,976 & 2,966 & 0.268 & 0.478\\
SGA & -20.896 & -27.575 & 4.5 & 2.9 & -21.030 & -27.416 & 51 & 51 & 0.237 & 0.541 \\
SOAPdenovo & -20.816 & -27.160 & {\bf 131.7} & 14.3 & -20.816 & -27.152 & 660 & 660 & 0.214 & 0.453\\
Velvet & -20.903 & -27.314 & 15.7 & 14.5 & -20.907 & -27.246 & 353 & 270 & 0.219 & 0.471\\
\emph{R. sphaeroides} ATCC 17025 & -29.391 & -29.973 & 3,218 & 3,218 & -29.391 & -29.973 & 3,218 & 3,218 & 0.813 & 0.904\\
\emph{R. capsulatus} & -29.953 & -29.997 & 3,739 & 3,739 & -29.953 & -29.997 & 3,739 & 3,739 & 0.978 & 0.995\\
\emph{truth} & -20.769 & -27.071 & 3,189 & 3,189 & -20.769 & -27.071 & 3,189 & 3,189 & 0.209 & 0.432\\
\hline
\end{tabular}
\caption[\emph{Rhodobacter sphaeroides} 2.4.1 assembly evaluation]{Assembly likelihood scores for \emph{Rhodobacter sphaeroides} 2.4.1 from the GAGE project~\cite{earl2011assemblathon}.  The results are presented
separately for the contigs and scaffolds and include the number of
unassembled reads (singletons), the LAP scores computed on unmated reads (LAP reads) or
mate-pairs (LAP mates), the N50 contig/scaffold sizes (N50),
and the reference-corrected N50 contig/scaffold sizes (CN50).
The best (maximum) value for each
genome-measure combination is highlighted in bold.
The results for the reference
assembly (either complete genome or high-quality draft) is given in the row
marked \emph{truth}.  In addition, we
provide the results for a closely related strain and species.
All values, except the LAP scores, were taken from the
GAGE publication. A threshold probability of 1e-30 was used for calculating the LAP scores. The standard deviations for the LAP's reads and LAP's mates scores are 0.00685 and 0.00969, respectively.}
\label{tab:rhodobacter}
\end{table}
\renewcommand{\baselinestretch}{2}
\small\normalsize
\end{landscape}



\begin{landscape}
\renewcommand{\baselinestretch}{1}
\small\normalsize
\begin{table*}[tb!]
\centering
\tiny
%\scalebox{0.50}{
\begin{tabular}{|c|c|c|c|c|c|c|c|c|c|c|}
\hline
 & \multicolumn{4}{c|}{Contigs} & \multicolumn{4}{c|}{Scaffolds} & &\\
\hline
Assembler & LAP reads & LAP mates & N50 (kb) & CN50 (kb) & LAP reads & LAP mates & N50 (kb) & CN50 (kb) & Unaligned reads (frac) & Unaligned mates (frac)\\
\hline
ABySS & {\bf -16.608} & -24.692 & 29.2 & 24.8 & {\bf -16.611} & -24.584 & 34 & 28 & {\bf 0.318} & 0.522\\
Allpaths-LG & -18.018 & -23.974 & 96.7 & {\bf 66.2} & -18.018 & {\bf -23.760} & 1,092 & {\bf 1,092} & 0.374 & {\bf 0.494}\\
Bambus2 & -18.083 & -24.256 & 50.2 & 16.7 & -18.085 & -23.899 & 1,084 & 1,084 & 0.375 & 0.503\\
MSR-CA & -18.282 & -24.258 & 59.2 & 48.2 & -18.282 & -23.926 & {\bf 2,412} & 1,022 & 0.389 & 0.508\\
SGA & -17.937 & -27.019 & 4 & 4 & -18.250 & -24.906 & 208 & 208 & 0.384 & 0.578\\
SOAPdenovo & -17.830 & {\bf -23.892} & {\bf 288.2} & 62.7 & -17.830 & -23.862 & 332 & 288 & 0.362 & 0.499\\
Velvet & -17.867 & -24.258 & 48.4 & 41.5 & -17.867 & -23.925 & 762 & 126 & 0.363 & 0.503\\
\emph{S. aureus} 04-02981 & -19.960 & -25.314 & 2,821 & 2,821 & -19.960 & -25.314 & 2,821 & 2,821 & 0.456 & 0.572\\
\emph{S. epidermidis} & -29.635 & -29.951 & 2,499 & 2,499 & -29.635 & -29.951 & 2,499 & 2,499 & 0.972 & 0.988\\
\emph{truth}  & -17.741 & -23.509 & 2,873 & 2,873 & -17.741 & -23.509 & 2,873 & 2,873 & 0.358 & 0.473\\
\hline
\end{tabular}
\caption[\emph{Staphylococcus aureus} USA300 assembly evaluation.]{Assembly likelihood scores for \emph{Staphylococcus
aureus} USA300 from the GAGE project~\cite{earl2011assemblathon}.
The results are presented
separately for the contigs and scaffolds and include the number of
unassembled reads (singletons), the LAP scores computed on unmated reads (LAP reads) or
mate-pairs (LAP mates), the N50 contig/scaffold sizes (N50),
and the reference-corrected N50 contig/scaffold sizes (CN50).
The best (maximum) value for each
genome-measure combination is highlighted in bold.
The results for the reference
assembly (either complete genome or high-quality draft) is given in the row
marked \emph{truth}.  In addition, we
provide the results for a closely related strain and species.
All values, except the LAP scores, were taken from the
GAGE publication. A threshold probability of 1e-30 was used for calculating the LAP scores. The standard deviations for the LAP's reads and LAP's mates scores are0.00740 and 0.0105, respectively.}
\label{tab:staph}
\end{table*}
\renewcommand{\baselinestretch}{2}
\small\normalsize
\end{landscape}

\begin{landscape}
\renewcommand{\baselinestretch}{1}
\small\normalsize
\begin{table}[tb!]
\centering
\tiny
\begin{tabular}{|c|c|c|c|c|c|c|c|c|c|c|c|}
\hline
 & \multicolumn{4}{c|}{Contigs} & \multicolumn{4}{c|}{Scaffolds} & & & \\
\hline
Assembler & LAP reads & LAP mates & N50 (kb) & CN50 (kb) & LAP reads & LAP mates & N50 (kb) & CN50 (kb) & CGAL Score & Unaligned reads (frac) & Unaligned mates (frac)\\
\hline
ABySS & -18.473 & -23.801 & 2 & 2 & -18.474 & -23.787 & 2.1 & 2 & -15.21 x $10^8$ &0.257 & 0.504\\
Allpaths-LG & -15.813 & -21.413 & 36.5 & 21 & -15.824 & -21.314 & {\bf 81,647} & {\bf 4,702} & -13.11 x $10^8$ &0.115 & 0.239\\
Bambus2 & -18.606 & -23.474 & 5.9 & 4.3 & -18.642 & -23.343 & 324 & 161 & - & 0.258 & 0.422\\
CABOG & {\bf -15.625} & {\bf -21.128} & {\bf 45.3} & {\bf 23.7} & {\bf -15.626} & {\bf -21.041} & 393 & 26 & {\bf -12.25 x} $\mathbf{10^8}$ & 0.109 & {\bf 0.229}\\
MSR-CA & -16.421 & -22.428 & 4.9 & 4.3 & -16.436 & -21.861 & 893 & 94 & - & 0.122 & 0.276\\
SGA & -15.712 & -22.990 & 2.7 & 2.7 & -16.909 & -22.326 & 83 & 79 & - & 0.134 & 0.328\\
SOAPdenovo & -15.702 & -21.705 & 14.7 & 7.4 & -15.734 & -21.594 & 455 & 214 & * & {\bf 0.101} & 0.269 \\
Velvet & -18.000 & -23.468 & 2.3 & 2.1 & -18.140 & -23.375 & 1,190 & 27 & - & 0.214 & 0.442\\
\emph{truth} & -15.466 & -21.001 & 107,349.50 & 107,349.50 & -15.466 & -21.002 & 107,349.50 & 107,349.50 & -11.25 x $10^8$ & 0.093 & 0.211 \\
\hline
\end{tabular}
\caption[\emph{Homo sapiens} chr 14 assembly evaluation]{Assembly likelihood scores for human chromosome 14 from the GAGE project~\cite{earl2011assemblathon} using a 10,000 read sample.
The results are presented
separately for the contigs and scaffolds and include the number of
unassembled reads (singletons), the LAP scores computed on unmated reads (LAP reads) or
mate-pairs (LAP mates), the N50 contig/scaffold sizes (N50),
and the reference-corrected N50 contig/scaffold sizes (CN50).
The best (maximum) value for each
genome-measure combination is highlighted in bold.
The results for the reference
assembly (either complete genome or high-quality draft) is given in the row
marked \emph{truth}.  In addition, we
provide the results for a closely related strain and species.
CGAL scores calculated from the long insert library were taken from the CGAL publication.
The authors only provided scores for the top three assemblies (Bowtie2 could not successfully map reads to the SOAPdenovo assembly).
All values, except the LAP and CGAL scores, were taken from the
GAGE publication. A threshold probability of 1e-30 was used for calculating the LAP scores. The standard deviation for both the LAP's reads and LAP's mates scores is 0.15.}
\label{tab:hg14}
\end{table}

\renewcommand{\baselinestretch}{2}
\small\normalsize
\end{landscape}


In the case of the human chromosome 14 assembly (Table~3), the scaffold-based
results do not agree with the reference-corrected contiguity values: the CABOG assembler outperforms Allpaths-LG in all but the
corrected scaffold N50 measure. This result highlights the inherent
difficulty of assessing the assembly quality even when a reference
sequence is available.  In this case, Allpaths-LG scaffold covers a
larger stretch of the genome; however, at the cost of errors both within the
contigs and in their relative placement.
Furthermore, the CABOG assembler is able to align nearly 0.1\% more mate-pairs than Allpaths-LG, despite having a far smaller scaffold size.

\begin{figure}[ht!]
\begin{center}
\includegraphics[width=4.86in]{figures/fig_4_asm1_ranks_reorder}
\end{center}
\renewcommand{\baselinestretch}{1}
\small\normalsize
\begin{quote}
\caption[Comparison between LAP scores and the rankings of the
    top assemblies generated in the Assemblathon~1 competition.]
    {Comparison between LAP scores and the rankings of the
    top assemblies generated in the Assemblathon~1 competition.
    The
    colors represent the relative ranking provided by the individual
    metrics (best - green, worst - red):
    log average probability (LAP),
    overall coverage (Cov tot),
    contig path NG50 (CPNG50),
    sum of all rankings from Assemblathon~1 (Overall),
    weighted median contig size based on estimated genome size (NG50),
    coverage within coding sequences (Cov genic),
    scaffold path NG50 (SPNG50),
    length for which half of any two valid columns in the assembly are correct in order and orientation (CC50),
    weighted median contig size based on total assembly size (N50),
    proportion of columns with a copy number error (Copy num),
    total substitution errors per correct bit (Subs),
    and sum of structural errors (Struct).
    Column descriptions and underlying data
    obtained from Table 3 in Earl et al.~\cite{earl2011assemblathon}.
    Columns are sorted according to the level of concordance with the
   LAP ranking.  \emph{De novo} measures are highlighted in bold.}
\label{fig:asm1_ranks}
\end{quote}
\end{figure}
\renewcommand{\baselinestretch}{2}
\small\normalsize

The Assemblathon 1 competition~\cite{earl2011assemblathon} further demonstrated the difficulty of accurately assessing the relative quality of genome
assemblies even when a correct reference sequence is available.
The authors developed a collection of quality metrics that measure the
stretch of a correctly assembled sequence (for example, contig path NG50 and scaffold
path NG50), the amount of structural errors (such as insertions,
deletions, and translocation), the long range contiguity (for example, the
average distance between correctly paired genomic loci), the number of copy number
errors, and the coverage within the assembly or only within coding
regions.  All these metrics were computed with respect to two
reference haplotypes, from which the read data were
simulated.  The authors ranked the different assemblies by each of the metrics and used the combined information to rank the
assemblies quality.

In Figure~\ref{fig:asm1_ranks}, we compare the rankings provided by our
LAP score to the rankings generated by the Assemblathon~1
competition. In addition to LAP, the figure also includes two variants
of the most commonly used \emph{de novo} measure of assembly size, N50
-- the weighted median contig size, that is, the length of largest contig $c$ such
that the total size of the contigs larger than $c$ exceeds half of the genome
size.  N50 uses the total assembly size as a proxy
for the genome size while the NG50 value uses a guess of the actual
genome size to compute the N50 value.  The more accurate estimation of the
genome size results in a better NG50's ranking, confirmed by the concordance with our LAP score.

The overall coverage measure (percentage of the reference haplotypes
covered by a particular assembly) correlates better with the LAP
score than the other metrics. This result is not surprising as the LAP score is strongly
affected by the number of the reads that can be correctly mapped to an
assembly, which is ultimately correlated with the concordance between
the assembly and the correct reference sequence.  Interestingly, the
overall rankings differ between LAP and the conclusions of the
Assemblathon~1 study.  Our analysis suggests that the BGI assembly is the best while the Assemblathon~1 picked the Broad
assembly as the winner.  This discrepancy can be partially explained in part by
the Broad's high performance within the genic regions (LAP does not
distinguish between genic and inter-genic segments) and the large
weight placed on the BGI's assembly's poor performance in terms of
substitution errors which have a relatively small effect on the LAP
score.

It is important to note that while LAP and the Assemblathon~1 results
disagree in the exact total ranking of the assemblies, the top 11 assemblies are
the same, meaning they are fundamentally of better quality than the
remaining 9 assemblies presented in the Assemblathon~1 paper.  In
fact, the Assemblathon overall score jumps from 74 for the 11th (WTSI-P)
assembly to 99 for the 12th (DCSISU) assembly, indicating a substantial qualitative
difference.  This is also reflected in the corresponding jump in the LAP score
from -37.326 to -39.441 for the 11th (DOEJGI) and 12th
(NABySS) assemblies, respectively.

\subsubsection{The effect of a contaminant DNA on the assessment of
  the assembly quality}
The Assemblathon~1 dataset provides an interesting challenge to the
assembly assessment.  The simulated libraries, generated in this
project from the human chromosome 13, also included approximately 5\%
of the contaminant DNA from an \emph{Escherichia coli} genome to
simulate commonly encountered laboratory contamination that possibly occur due to
the fragments of the cloning vector being sequenced along with the genome
of interest.  The participants to the Assemblathon~1 competition were
given the option to either remove the contaminant DNA prior to
assembly or retain the corresponding sequences in their assembly.
This decision has little effect on comparison between the resulting assembly
and the correct reference genome in the Assemblathon~1;
however, the ability of an assembler to correctly reconstruct the
contaminant genome significantly affects the corresponding LAP
score.

\begin{figure}[ht!]
\begin{center}
 \includegraphics[width=4.86in]{figures/fig_5_asm1_probs}
\end{center}
\renewcommand{\baselinestretch}{1}
\small\normalsize
\begin{quote}
\caption[Effect of a contaminant DNA on the computation of the LAP
    scores.]{Effect of a contaminant DNA on the computation of the LAP
    scores. Red crosses are the LAP scores computed on the entire read
    set (including contamination). Blue crosses are the LAP scores computed only on
    the `true' reads that map to the genome of interest. The
    corresponding LAP scores are quite similar (those obtained from a
    smaller set of reads are correspondingly smaller) except for those of assemblies that removed the contaminant DNA prior
    to assembly, and receive a boost in the LAP scores obtained on the ``true'' data.}
\label{fig:asm1_probs}
\end{quote}
\end{figure}
\renewcommand{\baselinestretch}{2}
\small\normalsize

Indeed, the LAP score (Figure~\ref{fig:asm1_probs})
computed from the entire set of reads (the red crosses) and that computed
after the contaminant reads were removed (the blue crosses) are strongly
correlated, the latter scores are slightly lower since they were computed on the smaller dataset.  In several cases, the assembly was performed
after removal of the contaminant DNA (see ``jumps'' in Figure~\ref{fig:asm1_probs}).  These
assemblies are penalized by our framework for not assembling the
contaminant DNA, a penalty that is removed once the same set of reads is
used for both assembly and quality assessment.

It is important to stress that the LAP
scores can only be meaningfully compared across the assemblies generated
from the same read set.  If a contaminant is known it should either be
removed from or retained within the dataset for all assemblers being
compared; otherwise, the corresponding scores can not be directly
compared. Note that this property is not unique to our measure:
ignoring or assembling contaminant DNA also affects other
traditional measures of quality, such as the N50 value or any
reference-based measures, for example, in the case where the contaminant DNA
shares significant similarity to the genome being assembled.

In practice, a `contaminant' is not known \emph{a
  priori}, and its definition depends on the specifics of an experiment.  In
general, it is difficult, if not impossible, to distinguish between
environmental contaminants and true artifacts in the data, both
in the context of metagenomic projects and in the case of
isolate genomes.  For example, the \emph{Bacillus anthracis} samples
from the bioterror attack in 2001, which were originally presumed to be uniform,
contained a mixture of very closely related strains, and the
characteristics of this mixture formed an important forensic marker in
the investigation~\cite{rasko2011bacillus}.

\begin{figure}[ht!]
\begin{center}
 \includegraphics[width=4.26in]{figures/fig_6_e_coli_ec10}
\end{center}
\renewcommand{\baselinestretch}{1}
\small\normalsize
\begin{quote}
\caption[Tuning SOAPdenovo k-mer parameter using LAP scores.]{Tuning SOAPdenovo k-mer parameter using LAP.
  LAP, N50, and corrected N50 are plotted for various SOAPdenovo assemblies of \emph{E. coli} K12 MG1655 dataset for different $k$-mer sizes (k=23-123). ALE~\cite{clark2013ale} scores are plotted alongside the LAP to show the differences between their underlying likelihood models.
  Also included is a breakdown of the errors along with the percentage of the unaligned reads for the various SOAPdenovo assemblies. Two vertical lines (at k=79 and k=87) correspond to the maximum ALE and LAP score, respectively.}
\label{fig:e_coli}
\end{quote}
\end{figure}
\renewcommand{\baselinestretch}{2}
\small\normalsize

\subsubsection{A useful application: tuning assembly parameters}
Our discussion so far has focused on comparing the output of different
assembly software with the goal of choosing the best assembler for a
particular dataset.  The developed probabilistic framework can
also be used to better choose the combination of parameters that allow
a particular assembly to achieve better results.  To demonstrate this
use case, we target the task of selecting the ``best'' (in terms of final assembly quality) $k$-mer length
for a de~Bruijn graph-based assembler.  We focus here on SOAPdenovo
assemblies of the \emph{Escherichia coli} K12 MG1655 genome (Figure~\ref{fig:e_coli}).

Without the availability of a
reference sequence, users of assembly software usually rely on
the N50 value as a proxy for the assembly quality.  In this case,
there is a clearly defined peak in N50 at k=79 (114,112 bp).
After adjusting for the assembly errors, there is a collection of the assemblies (k=47-51, 55-75) with nearly identical corrected N50s ($\sim$64,000 bp).
These assemblies range in N50 from $\sim$80-115 kbp.
Our \emph{de novo} measure LAP shows a clear peak at k=87, which corresponds to a corrected N50 of 59,352 bp.
It is important to note that despite roughly a 7\% difference from the peak in corrected N50 (k=63), the best LAP assembly contains 4 fewer indels larger than 5 bp, while also aligns roughly 54,000 more reads.

Alongside our LAP, we plot the likelihoods calculated from another assembly evaluator framework, ALE ~\cite{clark2013ale}.
The assembly with the highest ALE score (k=79) corresponds to the N50 peak.
Compared to the LAP selected assembly, the ALE selected assembly contains 10 more indels larger than 5 bp and has a 49\% drop from N50 to corrected N50 compared to the 35\% drop between those values for the LAP's selected assembly.


%%%%%%%%%%%%%%%%
%% Discussion %%
%%
\section{Discussion}
In this chapter, we have proposed a paradigm for the \emph{de novo} evaluation
of genome assemblies.  While the general paradigm could, in principle,
be used to provide an objective score of assembly quality, our
practical implementation of this paradigm, called the Log Average
Probability (LAP), is dataset specific and should only be used to
provide relative rankings of different assemblies of the same
dataset. Unlike traditional measures of assembly
contiguity (such as the N50 value), our reference-independent LAP
scores correlate with reference-based measures of assembly
quality.

We would like to stress that \emph{de novo} measures of assembly
quality, such as ours, are critically needed by researchers targeting
an assembly of yet unknown genomes.  The specific
characteristics of the data being assembled have a significant impact
on the performance of genome assemblers (in the Assemblathon~1~\cite{earl2011assemblathon} and GAGE~\cite{salzberg2011gage}
competitions, for example, different assemblers `won' the competition
depending on the analyzed dataset); thus, the reference-based quality
assessments cannot be reliably generalized to new genome projects.

In this chapter, we have made a number of simplifying assumptions for
modeling the sequencing process; specifically, that the sequencing
process is uniform (both in the coverage, and the error profile), and that the
reads are independently sampled from the genome (with the exception of
the dependence imposed by mate-pair experiments).
While our approach can detect copy number differences (unless the entire genome is exactly duplicated), it is with the caveat that sequencing biases within repetitive regions can possibly mask mis-assemblies.
More precise models
of the sequencing process that relax these assumptions can be easily
incorporated into our framework (e.g., effects of G/C content on
sequencing depth, or technology-specific error profiles). We plan
to create technology-specific variants of our score to keep up with the rapid changes in the characteristics of the sequencing data as new instruments and/or chemistries become available.
Furthermore, the probabilistic framework presented here can be used to
incorporate other types of information on the assembly quality,
for example, optical mapping data~\cite{zhou2007validation}.

\edit{In our assembler parameter-tuning experiment, we generated assemblies of \emph{Escherichia coli} K12 MG1655 using every allowed k-mer value. While this approach may be computationally feasible for smaller genomes, it is inefficient for very large, complex genomes.  One solution would be to use an optimization strategy for selecting potential k-mer values, e.g., with simulated annealing.
}

\edit{While there are differences between the LAP score and recent likelihood-based metrics, ALE and CGAL, these differences are quite small (Table~3 and Figure~\ref{fig:e_coli}).
Thus, it is important to discuss the technical improvements over ALE and CGAL.}
ALE's score did not perform quite as well as our LAP score on the parameter tuning experiment, and
CGAL is unable to evaluate all of the GAGE assemblies due to the technical limitations of Bowtie 2.
%Since CGAL relies on Bowtie 2 to align the sequences to the assembly, technical limitations of Bowtie 2 prevent CGAL from evaluating a few of the GAGE assemblies.
Bowtie 2 was not designed for reporting \emph{all} read alignments, which makes it very slow on large genomes. This problem will become more prevalent as sequencing costs continue to decrease, allowing for more complex genomes to be sequenced and assembled.
\edit{Our framework overcomes CGAL's limitations by allowing users to calculate the LAP score via the dynamic programming method on a subset of the reads or by using the SAM file produced from a read alignment tool designed for finding all alignments (e.g., mrsFAST\cite{hach2010mrsfast}).}

Our original goal was not to detect assembly errors, but to provide a global measure of how good an assembly may be.
We plan to extend our framework to detect assembly errors by adopting a similar approach to that demonstrated by ALE.

It is important to note that we have focused on a very specific use
case for assembly -- the complete reconstruction of a given genome.
Assembly algorithms are used in a number of other biological
applications, whose specific characteristics affect the validation of
the resulting assembly.  For example, studies targeting the genic
regions of an organism may tolerate large-scale rearrangements as long
as the individual genes are correctly reconstructed. In this context,
the validation framework would penalize substitution errors and small
insertions or deletions (which potentially affect gene structure) more
than mis-joins within intergenic regions.  Such application specific
tuning is possible within the proposed overall framework, and we
envisage the creation of a collection of community-supported modules
that compute application-specific LAP scores.

Our discussion has focused on the assembly of single genomes, however
the LAP score, as described, can also be directly used in the context
of diploid genomes or metagenomic mixtures.  In this case, our
score implicitly assumes that the goal of the assembler is to
correctly reassemble both the sequence and the relative abundances of the
individual haplotypes.  Assume, for example, a simple metagenomic
sample that contains two organisms; one that is twice as abundant as
the other one.  An assembler that produces three sequences,
corresponding to the three `haplotypes' in the sample (whether
explicitly outputting two, perhaps identical, versions of the abundant
organism or reporting the copy-number difference in some other way)
would obtain a better LAP score than an assembler that only reported
two sequences without any indication of their relative abundance. As
a result, the majority of metagenomic assemblers available today,
which only output the consensus sequence and not the relative
abundance of the contigs, would score poorly under our score.  We hope
that our work will inspire the developers of future metagenomic
assemblers to also output information on the relative abundance of the
reconstructed sequences, information that is critical to the analysis
of the data, yet rarely reported by existing tools.

Finally, we propose that measures such as ours, which objectively
capture the fit between the data being assembled and the output
produced by the assembler without relying on curated reference data
sets, become a standard tool in evaluating and comparing assembly
tools, allowing the community to move beyond simplistic measures of
contiguity such as the ubiquitous N50 measure.


% \bigskip
%
% %%%%%%%%%%%%%%%%%%%%%%%%%%%%%%%%
% \section{Competing Interests}
% The authors declare that they have no competing interests.
%
% \section{Author's contributions}
% MG and CH developed the LAP software. CH carried out the GAGE and Assemblathon1 experiments. CH and DS performed the assembler parameter tuning experiments. MG, CH, IA, and HL  developed the underlying statistical theory. MP conceived of the study, and participated in its design and coordination and helped to draft the manuscript. All authors read and approved the final manuscript.
%
%
% %%%%%%%%%%%%%%%%%%%%%%%%%%%
% \section{Acknowledgements}
% We would like to thank H\'{e}ctor Corrada Bravo and Bo Liu for their advice on the sampling procedure and associated statistics, Todd Treangen for advice on accessing the GAGE data, and the other members of the Pop lab for valuable discussions on all aspects of our work.
%
% This work was supported in part by the National Science Foundation
% (grants IIS-0812111, IIS-1117247 to MP), and by the National
% Institutes of Health (grant R01-HG-004885 to MP).

%%%%%%%%%%%%%%%%%%%%%%%%%%%%%%%%%%%%%%%%%%%%%%%%%%%%%%%%%%%%%
%%                  The Bibliography                       %%
%%                                                         %%
%%  Bmc_article.bst  will be used to                       %%
%%  create a .BBL file for submission, which includes      %%
%%  XML structured for BMC.                                %%
%%  After submission of the .TEX file,                     %%
%%  you will be prompted to submit your .BBL file.         %%
%%                                                         %%
%%                                                         %%
%%  Note that the displayed Bibliography will not          %%
%%  necessarily be rendered by Latex exactly as specified  %%
%%  in the online Instructions for Authors.                %%
%%                                                         %%
%%%%%%%%%%%%%%%%%%%%%%%%%%%%%%%%%%%%%%%%%%%%%%%%%%%%%%%%%%%%%


%%%%%%%%%%%%%%%%%%%%%%%%%%%%%%%%%%%
%%                               %%
%% Figures                       %%
%%                               %%
%% NB: this is for captions and  %%
%% Titles. All graphics must be  %%
%% submitted separately and NOT  %%
%% included in the Tex document  %%
%%                               %%
%%%%%%%%%%%%%%%%%%%%%%%%%%%%%%%%%%%

%%
%% Do not use \listoffigures as most will included as separate files






%%%%%%%%%%%%%%%%%%%%%%%%%%%%%%%%%%%
%%                               %%
%% Tables                        %%
%%                               %%
%%%%%%%%%%%%%%%%%%%%%%%%%%%%%%%%%%%

%% Use of \listoftables is discouraged.
%%
%\section{Tables}

% \begin{landscape}
% \renewcommand{\baselinestretch}{1}
% \small\normalsize

% \renewcommand{\baselinestretch}{2}
% \small\normalsize
% \end{landscape}
