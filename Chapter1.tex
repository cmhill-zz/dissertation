%Chapter 1

\newcommand{\edit}[1]{\textcolor{black}{#1}}

\renewcommand{\thechapter}{1}

\chapter{Introduction}



\section{Genome Assembly}

\subsection{Computational Challenges of Assembly}


\section{Contributions of This Dissertation}

In Chap.\ 2, we present the results of a computational study of the
influence of stochasticity on the dynamical evolution of multiple
four-wave-mixing processes in a single mode optical fiber with spatially
and temporally $\delta$-correlated phase noise. A generalized nonlinear
Schr\"odinger equation (NLSE) with stochastic phase fluctuations along the
length of the fiber is solved using the Split-step Fourier method
(SSFM). Good agreement is obtained with previous experimental and
computational results based on a truncated-ODE (Ordinary Differential
Equation) model in which stochasticity was seen to play a key role in
determining the nature of the dynamics. The full NLSE allows for
simulations with high frequency resolution (60\,MHz) and frequency span (16
THz) compared to the truncated ODE model (300\,GHz and 2.8\,THz,
respectively), thus enabling a more detailed comparison with
observations. A physical basis for this hitherto phenomenological phase
noise is discussed and quantified.

In Chap.\ 3, we discuss the implications of spontaneous and stimulated
Raman scattering on the project discussed in Chap.\ 2, namely, the dynamical evolution of
stochastic four-wave-mixing processes in an optical fiber.
The following question is asked - can stimulated Raman scattering be a mechanism by which
adequate multiplicative stochastic phase fluctuations are introduced in the
electric field of light undergoing four-wave-mixing as? Adequately checked numerical
algorithms of stimulated Raman scattering (SRS), spontaneous Raman generation and intrapulse
Raman scattering (IRS) are used while exploring this issue. The algorithms are described in detail, as also are
the results of the simulations. It is found that a 50-meter length of fiber (as used in the experiments),
is too short to see the influence of Raman scattering, which is found to eventually
dominate for longer fiber lengths.

In Chap.\ 4, self- and cross-phase modulation (XPM) of femtosecond pulses ($\sim$ 810
nm) propagating through a birefringent single-mode optical fiber ($\sim$ 6.9
cm) is studied both experimentally (using GRENOUILLE - Grating Eliminated
No Nonsense Observation of Ultrafast Laser Light Electric Fields)
%(using second harmonic
%generation-frequency resolved optical gating or SHG-FROG)
and numerically
(by solving a set of coupled nonlinear Schr\"odinger equations or
CNLSEs). An optical spectrogram representation is derived from the
electric field of the pulses and is linearly juxtaposed with the
corresponding optical spectrum and optical time-trace. The effects of
intrapulse Raman scattering (IRS) are discussed and the question whether
it can be a cause of asymmetric tranfer of pulse energies towards longer
wavelengths is explored. The simulations are shown to be in good qualitative
agreement with the experiments. Measured input pulse asymmetry, when incorporated
into the simulations, is found to be the dominant cause of output spectral
asymmetry. \renewcommand{\baselinestretch}{1} \small\footnotesize
\footnote{These averages are reported
for $45$ `detailed occupational codes', which is an intermediate
occupational classification (between two and three-digit codes)
given by the Current Population Survey (CPS).}
\renewcommand{\baselinestretch}{2} \small\normalsize
The results indicate that it is possible to modulate short pulses both temporally and spectrally by passage through polarization maintaining
optical fibers with specified orientation and length. The modulation technique is very direct and straightforward. No frequency components of the broadband pulse have to be rejected as the entire spectrum is uniformly modulated. The technique is flexible as the modulation spacing can be varied by varying the fiber length.

Chapter 5 provides the conclusion to the thesis.
