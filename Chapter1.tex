%Chapter 1

\newcommand{\edit}[1]{\textcolor{black}{#1}}

\renewcommand{\thechapter}{1}

\chapter{Introduction}

The genome of an organism is a blueprint for life.

The human genome was published in 2001.

Even a genome as highly researched as the human periodically releases updates correcting mistakes.
The main reference has undergone over 38 revisions.
Filling in gaps.
Correcting misassemblies.
When a new organsim is assembled, it is not easy determining what is missing, what is a mistake, and what is experimental artifact.



\section{Genome Assembly}

\subsection{Computational Challenges of Assembly}


\section{Contributions of This Dissertation}

In Chapter 2, we present an objective and holistic approach for evaluating and
comparing the quality of assemblies derived from a same dataset.  Our
approach defines the quality of an assembly as the likelihood that the
observed reads are generated from the given assembly, a value which can
be accurately estimated by appropriately modeling the sequencing
process. We show that our approach is able to automatically and accurately reproduce the
reference-based ranking of assembly tools produced by highly-cited assembly competitions: the
Assemblathon~\cite{earl2011assemblathon} and GAGE~\cite{salzberg2011gage}
competitions.

In Chapter 3, we extend our \emph{de novo} LAP framework to evaluate metagenomic assemblies.
We will show that by modifying our likelihood calculation to take into account abundances of assembled sequences, we can accurately and efficiently compare metagenomic assemblies.
%We evaluate our extended framework on results generated from the Human Microbiome Project (HMP) and
We find that our extended LAP framework is able to reproduce results on data from the Human Microbiome Project (HMP) that closely match the reference-based evaluation metrics and outperforms other \emph{de novo} metrics traditionally used to measure assembly quality.
Finally, we have integrated our LAP framework into the metagenomic analysis pipeline MetAMOS, allowing any user to reproduce quality assembly evaluations with relative ease.

In Chapter 4, we provide a novel regression testing framework for genome assemblers.
Our framework that uses two assembly evaluation mechanisms: \emph{assembly likelihood}, calculated using LAP\cite{LAP}, and \emph{read-pair coverage}, calculated using REAPR\cite{hunt2013reapr}, to determine if code modifications result in non-trivial
changes in assembly quality.
%We evaluate our framework using popular assemblers SOAPdenovo \cite{li2010novo} and Minimus \cite{sommer2007minimus}.
We study assembler evolution in two contexts. First,
we examine how assembly quality changes
throughout the version history of the popular assembler SOAPdenovo. Second,
we show that our framework can correctly evaluate decrease in assembly quality
using fault-seeded versions of another assembler Minimus.
Our results show that our framework accurately detects trivial
changes in assembly quality produced from permuted input reads and using
multi-core systems, which fail to be detected using traditional regression
testing methods.

In Chapter 5, we build on the pipeline described in Chapter 4 and introduce VALET, a \emph{de novo} pipeline for finding mis­assemblies within metagenomic assemblies.
We flag regions of the genome that are statistically inconsistent with the data generation process and underlying species abundances.
VALET is the first tool to accurately and efficiently find mis­assemblies in metagenomic data sets.
We provide a detailed list of predicted mis­assemblies in the Human Microbiome Project and use these findings to suggest improvements for future metagenomic assemblers.

In Chapter 6, we discuss our other contributions to bioinformatics relating to the domains of clustering, compression, and cloud computing.
