%Abstract Page 

\hbox{\ }

\renewcommand{\baselinestretch}{1}
\small \normalsize

\begin{center}
\large{{ABSTRACT}} 

\vspace{3em} 

\end{center}
\hspace{-.15in}
\begin{tabular}{ll}
Title of dissertation:    & {\large  BEYOND GENOME ASSEMBLY: }\\
&				      {\large COMPARING, VALIDATING, AND } \\
&                     {\large THIRD-GENERATION SEQUENCING} \\
\ \\
&                          {\large  Christopher Michael Hill, Doctor of Philosophy, 2015} \\
\ \\
Dissertation directed by: & {\large  Professor Mihai Pop} \\
&  				{\large	 Department of Computer Science } \\
\end{tabular}

\vspace{3em}

\renewcommand{\baselinestretch}{2}
\large \normalsize

FILLER
The current revolution in genomics has been made possible by software tools called genome assemblers, which stitch together DNA fragments “read” by sequencing machines into complete or nearly complete genome sequences. Despite decades of research in this field and the development of dozens of genome assemblers, assessing and comparing the quality of assembled genome sequences still relies on the availability of independently determined standards, such as manually curated genome sequences, or independently produced mapping data.

In the first part of my talk, I will describe our de novo  probabilistic measure of assembly quality which allows for an objective comparison of multiple assemblies generated from the same set of reads. I will detail extensions to our probabilistic framework that allows for an accurate evaluatation of metagenomic assemblies in addition to single genomes.

In the second part of my talk, I will describe VALET, a de novo pipeline for finding metagenomic mis-assemblies.  We flag regions of the genome that are statistically inconsistent with the data generation process and underlying species abundances. VALET is one of the first de novo tools to accurately and efficiently find mis-assemblies in metagenomic datasets and is currently being used to evaluate assemblies in the The Critical Assessment of Metagenome Interpretation (CAMI) competition.

For the final part of my talk, I will discuss my ongoing work with long read sequencing technologies. Long read sequencing technologies have brought us closer to the goal of a complete genome assembly.  The first computationally difficult step in most assembly algorithms is identifying sequences that overlap.  Here, we propose an efficient filtering method relying on SPQR tree-based decomposition that allows us to provide a locality sensitive labeling for these long, high-error reads.  In addition to providing us with a more efficient assembly, the tree-based decomposition of the assembly graph allows us to uncover population variatints when with multiple samples.
